\documentclass[spanish, fleqn]{article}
\usepackage[spanish]{babel}
\usepackage[utf8]{inputenc}
\usepackage[T1]{fontenc}
\usepackage[a4paper,bindingoffset=0.0in,left=0.75in,right=0.85in,top=0.9in,bottom=0.90in,footskip=.25in]{geometry}

%% Para las cosas bonitas en las esquinas de cada página
\usepackage{fancyhdr}

\usepackage[colorlinks, urlcolor=blue]{hyperref}

%% Estilo personalizado del abstract
\usepackage{abstract}
\usepackage{xcolor}

\renewenvironment{abstract}{{\normalfont\textbf{Resumen\\}}}{}

\definecolor{abstractGray}{RGB}{230,230,230}

\usepackage{mdframed}
\mdfdefinestyle{mdfabstract}{%
    %linecolor=acsblue, linewidth=0.8pt,
    backgroundcolor=abstractGray,
    leftline=false, rightline=false,
    topline=false, bottomline=false
    %innertopmargin=0.25cm, innerbottommargin=0.25cm,
    %innerleftmargin=0.25cm, innerrightmargin=0.25cm,
}


\usepackage{amsmath, amssymb} %Symbols
\usepackage{dsfont} %Symbols
\usepackage{multicol}

%\usepackage{graphicx}

%\usepackage{verbatim}
%\usepackage{listings}

\newcommand{\sigla}{INF-295}
\newcommand{\ramo}{Inteligencia Artificial}
\newcommand{\tarea}{Estado del Arte}
\newcommand{\nombreTarea}{Milk Collection with Blending}

\begin{document}
%% Portada.

\title{ \sigla: \ramo \\ \tarea: \nombreTarea }
\author{\href{mailto:anghelo.carvajal.14@sansano.usm.cl}{Anghelo Carvajal} \\ 201473062-4}
\date{\today}

\maketitle

%% Desactiva la numeración de páginas
\pagenumbering{gobble} 

%--------------------No borrar esta secci\'on--------------------------------%
\section*{Evaluación}

\begin{center}
    
    \begin{tabular}{ll}
    Resumen (5\%): & \underline{\hspace{2cm}} \\
    Introducción (5\%):  & \underline{\hspace{2cm}} \\
    Definición del Problema (10\%):  & \underline{\hspace{2cm}} \\
    Estado del Arte (35\%):  & \underline{\hspace{2cm}} \\
    Modelo Matemático (20\%): &  \underline{\hspace{2cm}}\\
    Conclusiones (20\%): &  \underline{\hspace{2cm}}\\
    Bibliografía (5\%): & \underline{\hspace{2cm}}\\
     &  \\
    \textbf{Nota Final (100\%)}:   & \underline{\hspace{2cm}}
    \end{tabular}

\end{center}

%---------------------------------------------------------------------------%


%% Portada end.

\newpage

% Estilo de las esquinas de cada página
\pagestyle{fancy}
\fancyhf{}
\lhead{\nombreTarea}
\rhead{\ramo}
\lfoot{\LaTeXe}
\rfoot{Página \thepage}

%% Activa numeración de páginas
\pagenumbering{arabic} 

\vspace{-16cm}

\begin{mdframed}[style=mdfabstract]
\begin{abstract}
Este trabajo investigativo presenta el actual estado del arte del \textit{Irregular Strip Packing Problem}, el cual es una variante del \textit{Strip Packing Problem}, con la diferencia de que las figuras a colocar son irregulares, pero se mantiene intacto el concepto de una cinta de ancho fijo y largo a ser minimizado. Se definirá el problema, mostrando variables y restricciones típicas, y se mostrará un modelo matemático que modela el problema de forma entendible y sencilla. 
\end{abstract}
\end{mdframed}


\vspace{8pt}


\begin{multicols}{2}
\section{Introducción}

El presente documento tiene como propósito presentar y definir el problema \textit{Milk Collection with Blending}, además de documentar el actual estado del arte de dicho problema.

Se comenzará definiendo el problema actual, hablando de forma general de las variables y restricciones típicas de este problema. Luego se expondrá el actual estado del arte del problema \textit{Milk Collection with Blending}, documentando sus orígenes (de que tipo de problema proviene), como han sido enfrentados dichos orígenes, los algoritmos que intentan resolverlo, y las variantes de este mismo problema. 

También se presentará un modelo matemático de programación lineal que puede representar este problema de forma certera. Finalmente se expondrán las conclusiones de este trabajo investigativo.

%\begin{multicols}{2}
\section{Definición del Problema}

El problema \textit{Milk Collection with Blending} es un problema NP-duro, el cual surge en el año 2016 \cite{MilkWithBlending}.

Este problema consiste en encontrar un conjunto de rutas óptimas para el recorrido de cada uno de los camiones que se tienen a disposición. Cada una de estas rutas debe proveer un camino para cada camión, de modo que este recoja toda la leche de cada una de las granjas que se le asigne y la lleve a la planta procesadora.

Cada ruta puede empezar en cualquier parte, pero siempre debe terminar en la planta procesadora. Hay un costo asociado al desplazamiento de un camión entre una granja y la otra.

La leche se categoriza en distintos tipos según su calidad. Estos tipos son ordenables de mejor a peor calidad, y las ganancias monetarias también son distintas según dicha calidad.
La planta procesadora exige una cantidad mínima de cada tipo (calidad) de leche.

Los camiones pueden transportar una cantidad limitada y no tienen compartimientos separados para cada tipo de leche, por lo que si un camión recoge leches de distintas calidades de las granjas, estas se mezclan dentro del camión, resultando en leche que se considera de la peor calidad de la mezcla. La ventaja de esto es reducir el costo de movilización de los camiones a cambio de menores ganancias por la calidad de la leche.

Los parámetros del problema son los caminos entre las granjas de leche (representadas por un grafo), los caminos entre las granjas y la planta, y el costo de desplazar a un camión a través de cada uno de estos arcos. La cantidad de camiones de las que se dispone y la capacidad que cada uno puede transportar, las clasificaciones para los tipos de leche, que tipo y cuanta cantidad de leche produce cada granja. Cuanta cantidad de leche de cada tipo exige la planta procesadora en total. Granja en que cada camión inicia su ruta.

Las variable principal de este problema es el orden en el cual cada camión recorre sus granjas, lo cual implica que tipos de leche recogería dicho camión de cada granja, que tipo de leche resultante entrega el camión a la planta y en que cantidad.

Este problema se ve restringido por la capacidad máxima que tiene cada camión, que cada camión recolecte toda la leche de la granja a la que va a recolectar, que cada granja sea visitada por a lo más un camión, cada camión tiene a lo más 1 ruta, se debe respetar la cantidad de leche de cada tipo que exige la planta como mínimo.

El objetivo de este problema es maximizar los beneficios monetarios, disminuyendo los recorridos de los camiones y aumentando la ganancia producida por la leche recolectada según su tipo.

También existen otras variantes de este problema, como que cada camión pueda poseer distintos compartimientos para transportar la leche \cite{memeticalgorithmtabusearch}, de modo que 1.- no habría mezcla de leches o 2.- que se minimice la mezcla de tipos de leche; que existan puntos de recolección a los cuales las granjas acercan la leche \cite{MilkBlendingPoints}; o que algunas granjas no sean accesibles por grandes camiones \cite{MilkWithIncompatibilityConstraints}.

%\end{multicols}

%\newpage

%\begin{multicols}{2}
\section{Estado del Arte}

Este problema es una variación del problema \textit{Vehicle Routing Problem} (VRP), el cual fue documentado por primera vez en el año 1959 \cite{TruckDispatchingProblem}. En dicho problema se discutía el encontrar un conjunto de rutas óptimas para una determinada cantidad de vehículos los cuales deben entregar paquetes a los respectivos clientes.

Específicamente este problema no ha sido tan trabajado e investigado debido a ser un problema no tan antiguo. Algunos de los problemas similares y métodos confeccionados para resolver esos y este problema son:

\begin{itemize}
    \item El primer VRP fue realizado y documentado por Dantzig y Ramser en 1959 \cite{TruckDispatchingProblem}. Este se basaba en un problema de distribución de combustible, y apuntaba a repartir paquetes a clientes geográficamente separados usando un conjunto de vehículos. Aquí se concluye que este tipo de problema es NP-completo, debido a que podía ser reducido a un problema de vendedor viajero.
    
    \item A la fecha, la implementación que mejores resultados ha dado es la de Taillard \cite{Paralleliterativesearchroutingproblems}, la cual se basa en algoritmos de búsqueda tabú, en donde distribuye a los nodos en dos formas distintas, uno de forma uniforme y la otra de forma no euclidiana.
    
    \item Otro tipo de problema basado en VRP es la variante con múltiples productos (MPVRP), el cual ha sido enfrentado con algoritmos genéticos \cite{Modellingtransportlogisticssegregation}, el algoritmo de Dijkstra \cite{multiproductpackingdelivery} y modelos de programación lineal entera en conjunto a un solver \cite{optimizationvehiclethreedimensional} por nombrar algunos.
    
    \item Una variante similar a MPVRP es el problema de enrutamiento de vehículos con múltiples compartimientos (MCVRP). Un método notable es el de El Fallahi et. al. \cite{memeticalgorithmtabusearch}, dado que lo resuelven de tres formas distintas; con una heurística constructiva sin iteraciones de mejora, con busqueda tabú y con un algorítmo memético (el cual es a su vez una extensión del los algoritmos genéticos).
    
    \item La primera vez que se atacó de la recolección de leche como una variante específica de VRP fue en el año 1994 por Sankaran y Ubgade \cite{firstmilk} para resolver un caso real de 70 granjas en India. Ellos consideraron camiones de diferentes capacidades y no poder exceder ciertos límites de tiempo entre cada recolección dado a las distintas condiciones climáticas de la zona.
    
    \item El problema de la recolección de leche que admite la mezcla de los distintos tipos de leche (nuestro problema) fue abordado por primera vez el año 2016 por Germán Paredes-Belmar et. al \cite{MilkWithBlending}. En dicha ocasión se formuló el problema a través de un modelo entero mixto. Para resolver instancias medianas usaba un algoritmo de bifurcación y corte. Para instancias más grandes se utilizaba un procedimiento heurístico, el cual consiste en dividir la instancia real en conjuntos de granjas, las cuales eran repartidas según su ubicación. En este estudio se discute el mezclar los tipos de leches vs no mezclarlos al transportarlos en los camiones, y concluía que mezclar las leches predominaba por sobre no mezclarlos, debido a la viabilidad y relajación que esta otorga. También estudia como afecta los camiones con múltiples compartimientos y con compartimiento único a esta variante.
    
    \item En el año 2017, los mismos autores enfrentan una variante del problema \cite{MilkBlendingPoints}. En este problema se tenía una cantidad aún mayor de granjas, por lo que el modelo anterior no podía encontrar una solución en tiempo razonable, por lo que agregaron puntos de recolección para cada conjunto de granjas, lo cual permitió encontrar soluciones en tiempos mucho más acotados.

    \item Villagran propone 2 acercamientos, basados técnicas de búsqueda local, para este problema, en el año 2019 \cite{Lechemezclada2}. Estos algoritmos eran basados en las técnicas \textit{hill-climbing} e \textit{iterated local search}, las cuales ambas entregaron soluciones de buena calidad, de modo que abre la puerta a la posibilidad de no usar técnicas completas para la resolución de este problema. En las pruebas del autor, estos algoritmos lograron obtener el óptimo global en la mayoría de instancias utilizadas. Cabe destacar que ambas implementaciones entregaban mejores resultados, tanto en calidad de la solución como en tiempo de ejecución, que los algoritmos del estado del arte hasta esa fecha.
    
    \item En el año 2019, Soto \cite{LecheConMezclaAnnealing} propone un acercamiento basado en la meta-heurística \textit{Simulated Annealing}, el cual tiene un muy buen desempeño en instancias pequeñas y medianas, comparables al solver CPLEX.

\end{itemize}


%\end{multicols}

%\newpage
%\begin{multicols}{2}
\section{Modelo Matemático}

A continuación se presenta un modelo de programación entera mixta, tal como es descrito en \cite{Lechemezclada2}, el cual está basado en \cite{MilkWithBlending}.

\subsection{Función objetivo}

Este problema tiene como objetivo la maximización del beneficio monetario, por lo que se considera la diferencia entre la ganancia producida por la leche recolectada vs los costos de transporte de las rutas construidas.

Esto se representa a través de:

\begin{equation}
    Max \left\{ \sum_{t\in T}\sum_{r\in T} \alpha^{r}v^{tr} - \sum_{(i, j, k \in A K)} c_{i j}^{k}x_{i j}^{k} \right\}
\end{equation}


\subsection{Parámetros}

Los parámetros de este problema son los siguientes:

\begin{itemize}
    \item $A$: Conjunto de arcos que representan caminos entre productores de leche.
    \item $A^0$ : Conjunto de arcos que representan caminos entre planta y productores de leche.
    \item $N$: Conjunto de productores, $N = 0, ..., n$. Se consideran en total n productores.
    \item $N_0$ : Conjunto de productores y la planta.
    \item $K$: Conjunto de camiones
    \item $T$ : Conjunto de calidades de leche
    \item $N^t$ : Conjunto de productores de leche de calidad $ t \in T $.
    \item $D^t$ : Resultado de la mezcla de leche de calidad $r$ con leche de calidad $t$.
    \item $IT$ : Conjunto de pares ordenados $(i, t)$ de productores $i$ y leche de calidad $t$, donde cada cliente produce solo una calidad de leche.
    \item $Q^k$ : Capacidad de cada camión $k$.
    \item $q_i^t$ : Cantidad de leche $t$ producida por el productor $i$.
    \item $c_{i j}^{k}$ : Costo del viaje de cada camión $k$ sobre el arco $(i, j) \in A \cup A^0$.
    \item $\alpha^t$ : Ingreso por unidad leche de calidad $t$.
    \item $P^t$ : Requerimientos de leche de calidad $t$ de la planta.
\end{itemize}



\subsection{Variables de decisión}

Este modelo posee 5 variables de decisión. De estas, 3 son binarias:
\begin{itemize}
    \item $x_{i j}^k$, la cual vale 1 si el camión $k$ viaja directamente del nodo $i$ al nodo $j$. Espacio de búsqueda: $2^{|N| * |N| * |K|}$
    \item $y_{i}^{k t}$, la cual vale 1 si el camión $k$ recoge leche de calidad $t$ de la granja $i$. Espacio de búsqueda: $2^{|N| * |K| * |T|}$
    \item $z^{k t}$, la cual vale 1 si el camión $k$ entrega leche de calidad $t$ a la planta. Espacio de búsqueda: $2^{|K| * |T|}$
\end{itemize}

Las otras 2 variables no binarias:
\begin{itemize}
    \item $w^{k t}$, la cual indica el volumen de leche de calidad $t$ que el camión $k$ entrega a la planta. Espacio de búsqueda: $|Q|^{|K| * |T|}$
    \item $v^{t r}$, la cual indica el volumen de leche de calidad $t$ entregada a la planta, mezclada para su uso como leche de calidad $r$. Espacio de búsqueda: $|Q|^{|T| * |T|}$
\end{itemize}


\subsubsection{Espacio de búsqueda}

El espacio de búsqueda es:

\begin{equation}
    2^{|N| * |N| * |K|} * 2^{|N| * |K| * |T|}  * 2^{|K| * |T|} * |Q|^{|K| * |T|} * |Q|^{|T| * |T|}
\end{equation}

Lo cual puede ser reescrito como:

\begin{equation}
    2^{|K|*(|N|^2 + |N|*|T| + |T|)} * |Q|^{|T| * (|K| * |T|)}
\end{equation}

\subsection{Restricciones}

Este modelo se ve restringido por las siguientes restricciones:

%\begin{itemize}
     La restricción \ref{eq:1} limita la cantidad de leche que puede recolectar cada camión de
acuerdo a su capacidad. Se considera una flota heterogénea.
    
        \begin{align} \label{eq:1}
            \sum_{r\in T} \sum_{i \in N: (i, j) \in IT} q_{i}^{t} y_{i}^{k t} \leq Q^k, \forall k \in K
        \end{align}
        

     La restricción \ref{eq:2}  establece que la recolección de la leche de cada productor debe ser
realizada por exactamente un camión. Esto implica que se debe recolectar la leche de
todos los productores y que un productor no puede ser visitado más de una vez.

        \begin{align} \label{eq:2}
            \sum_{k\in K_i} y^{k t} = 1, \forall i \in N, t \in T : (i, j) \in IT
        \end{align}
        
     La restricción  \ref{eq:3} establece que cada camión debe tener como máximo una ruta la cual
comienza desde la planta.

        \begin{align} \label{eq:3}
            \sum_{j: (0_k, j, k) \in AK} x_{0_k j}^{k} \leq 1, \forall k \in K
        \end{align}
        
     La restricción \ref{eq:4} permite controlar el flujo para el orden de las visitas de los nodos
por parte de cada camión.

        \begin{align} \label{eq:4}
            \sum_{i: (i, j, k) \in AK} x_{i j}^{k} =  \sum_{h: (j, h, k) \in AK} x_{j h}^{k} , \forall k \in K_j, j \in N_0
        \end{align}

     La restricción \ref{eq:5} establece que cada camión que visita cada granja debe detenerse y
recoger su leche.

        \begin{align} \label{eq:5}
            \sum_{p: (p, i, k) \in AK} x_{p i}^{k} =  y_{i}^{k t} , \forall k \in K_i, i \in N, t \in T: (i, t) \in IT
        \end{align}

     Las restricciones \ref{eq:6}, \ref{eq:7}, \ref{eq:8} y \ref{eq:9} establecen las reglas de mezcla de leche. La restricción \ref{eq:6} controla el tipo de leche de cada camión de acuerdo a cada una de las granjas que ha visitado. La restricción \ref{eq:7} controla que cada camión entrega en planta solo un tipo de leche. La restricción \ref{eq:8} mide la cantidad de leche entregada de cada tipo de acuerdo a la capacidad de los camiones que recolectaron cada tipo de leche. Por último, la restricción \ref{eq:9} mide la cantidad efectivamente recolectada de cada tipo de leche considerando las granjas visitadas por cada camión.

        \begin{align} \label{eq:6}
            z^{k t} \leq 1 - \sum_{r \in D^t: r \neq t, (i, r) \in IT }  y_{i}^{k r} , \forall k \in K_i, i \in N, t \in T
        \end{align}
        \begin{align} \label{eq:7}
            \sum_{t\in T} z^{k t} \leq 1, \forall k \in K
        \end{align}
        \begin{align} \label{eq:8}
            w^{ k t } \leq z^{k t} Q^k, \forall k \in K, t \in T
        \end{align}
        \begin{align} \label{eq:9}
            w^{k t} \leq \sum_{r: t\in D^r} \sum_{h\in N^r} q_{h}^{r} y_{h}^{k r}  , \forall k \in K, t \in T
        \end{align}

     La restricción \ref{eq:10} se encarga de forzar que cada camión se lleve toda la leche producida por cada granja a la planta.

        \begin{align} \label{eq:10}
            \sum_{k\in K} \sum_{t\in T} w^{k t} = \sum_{(i, t) \in IT} {q_i^{t}}
        \end{align}


     La restricción \ref{eq:11} se encarga de equilibrar la cantidad de leche de cada calidad que
llega a la planta y la cantidad de leche de cada calidad restante después de la mezcla
en la planta.

        \begin{align} \label{eq:11}
            \sum_{r\in D^t} v^ {t r} = \sum_{k\in K} w^{k t} , \forall t \in T
        \end{align}

     La restricción \ref{eq:12} se encarga de controlar la satisfacción de las cuotas de leche de
cada tipo.

        \begin{align} \label{eq:12}
            \sum_{t\in T} v^{t r} \geq P^r , \forall r \in D^t
        \end{align}


     La restricción \ref{eq:13} evita mezclas de leche prohibidas.

        \begin{align} \label{eq:13}
            y_i^{k t} + y_i^{k r} \leq 1, \forall (t, r) \in PM; (i, t), (j, t) \in IT
        \end{align}


     La restricción \ref{eq:14} evita la aparición de sub-ciclos en las rutas de cada camión.

        \begin{align} \label{eq:14}
            \sum_{i \in S}\sum_{j \in S} x_{i j}^{k} \leq |S| -1 , \forall S \subseteq N, k \in K
        \end{align}


     Las restricciones \ref{eq:15}, \ref{eq:16} y \ref{eq:17} controlan la naturaleza de las variables de decisión del modelo planteado. La restricción \ref{eq:15} controla la naturaleza binaria de las variables asociadas a los tipos de leche recolectados y mezclados en ruta. La restricción \ref{eq:16} controla la naturaleza de la variable binaria que controla las secuencias de visitas de los camiones. La restricción \ref{eq:17} controla la naturaleza no negativa de las variables de volúmenes de leche entregados y mezclados en planta.

        \begin{align} \label{eq:15}
            y_{i}^{k t}, z^{k t} \in \{0, 1\}, \forall i\in N, k \in K_i, t\in T : (i, t) \in IT
        \end{align}
        \begin{align} \label{eq:16}
            x_{i j} ^ k \in {0, 1}, \forall (i, j, k) \in AK
        \end{align}
        \begin{align} \label{eq:17}
            w^{k t}, v^{t r} \geq 0 , \forall k \in K; t, r \in T, r\in D^t
        \end{align}

%\end{itemize}

Todas estas restricciones se trabajan como restricciones duras, es decir, toda solución al
problema debe cumplir con todas las restricciones impuestas.

%\end{multicols}
%\newpage

%\begin{multicols}{2}
\section{Conclusiones}

Como cualquier otro problema NP-duro, este problema CSOP ha demostrado ser difícil de modelar de modo que entregue una solución óptima en un tiempo prudente. 

A pesar de ser un problema joven (aunque SPP data de 1980, esta variante especifica data del 2006), se han desarrollado múltiples modelos y métodos de resolución (mayormente híbridos y basados en \textit{greedy}), los cuales han optimizado bastante los tiempos de ejecución para entregar resultados optimos.

\section{Bibliografía}

\bibliographystyle{plain}
\bibliography{referencias}

\end{multicols}

\end{document}
