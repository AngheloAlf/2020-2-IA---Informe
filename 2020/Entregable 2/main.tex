\documentclass[spanish, fleqn]{article}
\usepackage[spanish]{babel}
\usepackage[utf8]{inputenc}
\usepackage[T1]{fontenc}
\usepackage[a4paper,bindingoffset=0.0in,left=0.75in,right=0.85in,top=0.9in,bottom=0.90in,footskip=.25in]{geometry}

%% Para las cosas bonitas en las esquinas de cada página
\usepackage{fancyhdr}

\usepackage[colorlinks, urlcolor=blue]{hyperref}

%% Estilo personalizado del abstract
\usepackage{abstract}
\usepackage{xcolor}

\renewenvironment{abstract}{{\normalfont\textbf{Resumen\\}}}{}

\definecolor{abstractGray}{RGB}{230,230,230}

\usepackage{mdframed}
\mdfdefinestyle{mdfabstract}{%
    %linecolor=acsblue, linewidth=0.8pt,
    backgroundcolor=abstractGray,
    leftline=false, rightline=false,
    topline=false, bottomline=false
    %innertopmargin=0.25cm, innerbottommargin=0.25cm,
    %innerleftmargin=0.25cm, innerrightmargin=0.25cm,
}


\usepackage{amsmath, amssymb} %Symbols
\usepackage{dsfont} %Symbols
\usepackage{multicol}

%\usepackage{graphicx}

%\usepackage{verbatim}
%\usepackage{listings}

\newcommand{\sigla}{INF-295}
\newcommand{\ramo}{Inteligencia Artificial}
\newcommand{\tarea}{Informe final}
\newcommand{\nombreTarea}{Milk Collection with Blending}

\begin{document}
%% Portada.

\title{ \sigla: \ramo \\ \tarea: \nombreTarea }
\author{\href{mailto:anghelo.carvajal.14@sansano.usm.cl}{Anghelo Carvajal} \\ 201473062-4}
\date{\today}

\maketitle

%% Desactiva la numeración de páginas
\pagenumbering{gobble} 

%--------------------No borrar esta secci\'on--------------------------------%
\section*{Evaluación}

\begin{center}

    \begin{tabular}{ll}
    Mejoras 1ra Entrega (10 \%): &  \underline{\hspace{2cm}}\\
    Código Fuente (10 \%): &  \underline{\hspace{2cm}}\\
    Representación (15 \%):  & \underline{\hspace{2cm}} \\
    Descripción del algoritmo (20 \%):  & \underline{\hspace{2cm}} \\
    Experimentos (10 \%):  & \underline{\hspace{2cm}} \\
    Resultados (10 \%):  & \underline{\hspace{2cm}} \\
    Conclusiones (20 \%): &  \underline{\hspace{2cm}}\\
    Bibliografía (5 \%): & \underline{\hspace{2cm}}\\
    &  \\
    \textbf{Nota Final (100)}:   & \underline{\hspace{2cm}}
    \end{tabular}

\end{center}

%---------------------------------------------------------------------------%


%% Portada end.

\newpage

% Estilo de las esquinas de cada página
\pagestyle{fancy}
\fancyhf{}
\lhead{\nombreTarea}
\rhead{\ramo}
\lfoot{\LaTeXe}
\rfoot{Página \thepage}

%% Activa numeración de páginas
\pagenumbering{arabic} 

\vspace{-16cm}

\begin{mdframed}[style=mdfabstract]
    \begin{abstract}
    Este trabajo investigativo presenta el actual estado del arte del problema \textit{Milk Collection with Blending}, el cual es una sub-variante de \textit{Vehicle Routing Problem} (VRP). Este problema busca recolectar leche sin procesar de un conjunto de granjas, donde la mayor diferencia con VRP es que los distintos tipos de leche sin procesar pueden mezclarse entre si, lo cual produce una leche de calidad inferior. Se definirá el problema, mostrando variables y restricciones típicas, y se mostrará un modelo matemático que modela el problema de forma certera.
    \end{abstract}
\end{mdframed}


\vspace{8pt}


\begin{multicols}{2}
\section{Introducción}

El presente documento tiene como propósito presentar y definir el problema \textit{Irregular Strip Packing} (el cual consiste en la colocación de figuras irregulares en una cinta de ancho fijo de modo que se minimice el largo de esta cinta), y documentar el actual estado del arte de este problema. 

Se empezará definiendo el problema actual, hablando de forma general de las variables y restricciones típicas de este problema. Luego se expondrá el actual estado del arte del \textit{Irregular Strip Packing Problem} (ISPP), documentando métodos usados para resolverlo, las mejores representaciones y algoritmos hasta la fecha, heurísticas, tendencias y lo más importante que se ha hecho hasta ahora con relación al problema. Se presentará un modelo matemático simple, el cual es capaz de modelar el problema. Se opto por presentar un modelo simple a modo introductorio a los posibles modelos de este problema, con el inconveniente de que este modelo no es eficiente en comparación a otros. Finalmente se expondrán las conclusiones de este trabajo investigativo.



\section{Definición del Problema}

El problema \textit{Milk Collection with Blending} es un problema NP-duro, el cual surge en el año 2016 \cite{MilkWithBlending}.

Este problema consiste en encontrar un conjunto de rutas óptimas para el recorrido de cada uno de los camiones que se tienen a disposición. Cada una de estas rutas debe proveer un camino para cada camión, de modo que este recoja toda la leche de cada una de las granjas que se le asigne y la lleve a la planta procesadora.

Cada ruta puede empezar en cualquier parte, pero siempre debe terminar en la planta procesadora. Hay un costo asociado al desplazamiento de un camión entre una granja y la otra.

La leche se categoriza en distintos tipos según su calidad. Estos tipos son ordenables de mejor a peor calidad, y las ganancias monetarias también son distintas según dicha calidad.
La planta procesadora exige una cantidad mínima de cada tipo (calidad) de leche.

Los camiones pueden transportar una cantidad limitada y no tienen compartimientos separados para cada tipo de leche, por lo que si un camión recoge leches de distintas calidades de las granjas, estas se mezclan dentro del camión, resultando en leche que se considera de la peor calidad de la mezcla. La ventaja de esto es reducir el costo de movilización de los camiones a cambio de menores ganancias por la calidad de la leche.

Los parámetros del problema son los caminos entre las granjas de leche (representadas por un grafo), los caminos entre las granjas y la planta, y el costo de desplazar a un camión a través de cada uno de estos arcos. La cantidad de camiones de las que se dispone y la capacidad que cada uno puede transportar, las clasificaciones para los tipos de leche, que tipo y cuanta cantidad de leche produce cada granja. Cuanta cantidad de leche de cada tipo exige la planta procesadora en total. Granja en que cada camión inicia su ruta.

Las variable principal de este problema es el orden en el cual cada camión recorre sus granjas, lo cual implica que tipos de leche recogería dicho camión de cada granja, que tipo de leche resultante entrega el camión a la planta y en que cantidad.

Este problema se ve restringido por la capacidad máxima que tiene cada camión, que cada camión recolecte toda la leche de la granja a la que va a recolectar, que cada granja sea visitada por a lo más un camión, cada camión tiene a lo más 1 ruta, se debe respetar la cantidad de leche de cada tipo que exige la planta como mínimo.

El objetivo de este problema es maximizar los beneficios monetarios, disminuyendo los recorridos de los camiones y aumentando la ganancia producida por la leche recolectada según su tipo.

También existen otras variantes de este problema, como que cada camión pueda poseer distintos compartimientos para transportar la leche \cite{memeticalgorithmtabusearch}, de modo que 1.- no habría mezcla de leches o 2.- que se minimice la mezcla de tipos de leche; que existan puntos de recolección a los cuales las granjas acercan la leche \cite{MilkBlendingPoints}; o que algunas granjas no sean accesibles por grandes camiones \cite{MilkWithIncompatibilityConstraints}.


\section{Estado del Arte}

Este problema es una variación del problema \textit{Vehicle Routing Problem} (VRP), el cual fue documentado por primera vez en el año 1959 \cite{TruckDispatchingProblem}. En dicho problema se discutía el encontrar un conjunto de rutas óptimas para una determinada cantidad de vehículos los cuales deben entregar paquetes a los respectivos clientes.

Específicamente este problema no ha sido tan trabajado e investigado debido a ser un problema no tan antiguo. Algunos de los problemas similares y métodos confeccionados para resolver esos y este problema son:

\begin{itemize}
    \item El primer VRP fue realizado y documentado por Dantzig y Ramser en 1959 \cite{TruckDispatchingProblem}. Este se basaba en un problema de distribución de combustible, y apuntaba a repartir paquetes a clientes geográficamente separados usando un conjunto de vehículos. Aquí se concluye que este tipo de problema es NP-completo, debido a que podía ser reducido a un problema de vendedor viajero.
    
    \item A la fecha, la implementación que mejores resultados ha dado es la de Taillard \cite{Paralleliterativesearchroutingproblems}, la cual se basa en algoritmos de búsqueda tabú, en donde distribuye a los nodos en dos formas distintas, uno de forma uniforme y la otra de forma no euclidiana.
    
    \item Otro tipo de problema basado en VRP es la variante con múltiples productos (MPVRP), el cual ha sido enfrentado con algoritmos genéticos \cite{Modellingtransportlogisticssegregation}, el algoritmo de Dijkstra \cite{multiproductpackingdelivery} y modelos de programación lineal entera en conjunto a un solver \cite{optimizationvehiclethreedimensional} por nombrar algunos.
    
    \item Una variante similar a MPVRP es el problema de enrutamiento de vehículos con múltiples compartimientos (MCVRP). Un método notable es el de El Fallahi et. al. \cite{memeticalgorithmtabusearch}, dado que lo resuelven de tres formas distintas; con una heurística constructiva sin iteraciones de mejora, con busqueda tabú y con un algorítmo memético (el cual es a su vez una extensión del los algoritmos genéticos).
    
    \item La primera vez que se atacó de la recolección de leche como una variante específica de VRP fue en el año 1994 por Sankaran y Ubgade \cite{firstmilk} para resolver un caso real de 70 granjas en India. Ellos consideraron camiones de diferentes capacidades y no poder exceder ciertos límites de tiempo entre cada recolección dado a las distintas condiciones climáticas de la zona.
    
    \item El problema de la recolección de leche que admite la mezcla de los distintos tipos de leche (nuestro problema) fue abordado por primera vez el año 2016 por Germán Paredes-Belmar et. al \cite{MilkWithBlending}. En dicha ocasión se formuló el problema a través de un modelo entero mixto. Para resolver instancias medianas usaba un algoritmo de bifurcación y corte. Para instancias más grandes se utilizaba un procedimiento heurístico, el cual consiste en dividir la instancia real en conjuntos de granjas, las cuales eran repartidas según su ubicación. En este estudio se discute el mezclar los tipos de leches vs no mezclarlos al transportarlos en los camiones, y concluía que mezclar las leches predominaba por sobre no mezclarlos, debido a la viabilidad y relajación que esta otorga. También estudia como afecta los camiones con múltiples compartimientos y con compartimiento único a esta variante.
    
    \item En el año 2017, los mismos autores enfrentan una variante del problema \cite{MilkBlendingPoints}. En este problema se tenía una cantidad aún mayor de granjas, por lo que el modelo anterior no podía encontrar una solución en tiempo razonable, por lo que agregaron puntos de recolección para cada conjunto de granjas, lo cual permitió encontrar soluciones en tiempos mucho más acotados.

    \item Villagran propone 2 acercamientos, basados técnicas de búsqueda local, para este problema, en el año 2019 \cite{Lechemezclada2}. Estos algoritmos eran basados en las técnicas \textit{hill-climbing} e \textit{iterated local search}, las cuales ambas entregaron soluciones de buena calidad, de modo que abre la puerta a la posibilidad de no usar técnicas completas para la resolución de este problema. En las pruebas del autor, estos algoritmos lograron obtener el óptimo global en la mayoría de instancias utilizadas. Cabe destacar que ambas implementaciones entregaban mejores resultados, tanto en calidad de la solución como en tiempo de ejecución, que los algoritmos del estado del arte hasta esa fecha.
    
    \item En el año 2019, Soto \cite{LecheConMezclaAnnealing} propone un acercamiento basado en la meta-heurística \textit{Simulated Annealing}, el cual tiene un muy buen desempeño en instancias pequeñas y medianas, comparables al solver CPLEX.

\end{itemize}



\section{Modelo Matemático}

El modelo matemático propuesto en \cite{Leung2012LocalSearch} es:

\begin{itemize}
    \item La función objetivo a minimizar es $L(V, R)$.
    \item Los parámetros son:
    \begin{itemize}
        \item Un conjunto de polígonos $\mathcal{P} = \{\mathcal{P}_1, ..., \mathcal{P}_n\}$.
        \item Un conjunto de ángulos $\mathcal{O}$.
        \item El ancho $W$ de la cinta.
    \end{itemize}
    \item Las variables son:
    \begin{itemize}
        \item $r_i$: El ángulo de rotación para un polígono $i; 1 \leq i \leq n$. Su dominio es el conjunto de ángulos $\mathcal{O}$.
        \item $v_i$: Un vector de traslación ($v = (v_x, v_y)$) para el polígono $i; 1 \leq i \leq n$. Su dominio es $\mathds{R}^2$.
    \end{itemize}
    
    \item Algunas definiciones necesarias para el modelo:
    \begin{itemize}
        \item La cinta rectangular $\mathcal{C} = \mathcal{C}(W, L)$.
        
        \item Punto $p$ en el plano bidimensional, representado por $p = (p_x, p_y) \in \mathds{R}^2$.

        \item Segmento de linea $s$, el cual es un conjunto de puntos dentro de una limitada por 2 puntos $p_a$ y $p_b$. Representado por $s = \{p \in \mathds{R}^2, t\in [0, 1] \talque p = p_a + t (p_b) \}$.

        \item Polígono $P$, el cual es una figura plana encerrada por un camino cerrado compuesto por una secuencia finita de segmentos de linea (denotados $S$). No hay ningún cruce entre cualquier par de segmentos de linea en $S$. Esta representado por $P = \{p \in \mathds{R}^2 \talque f_S(p) = 1 \}$, donde $f_S(p)$ es la función de conteo $f_S(p) = |\{s' \in S \talque \exists x' < p_x: (x', p_y) \in s'\}|$
        
        \item Función de traslación $\oplus$, la cual toma un polígono $P$ y un vector de traslación $v$. Está se define como $P \oplus v = \{ (p_x' + v_x, p_y' + v_y) \talque p' \in P \}$
        
        \item Función de rotación, la cual recibe un polígono $P$ y un ángulo $r$. Está definida como $P(r) = \{(p_x'\cos(r) + p_y'\sin(r), \text{ } -p_x'\sin(r) + p_y'\cos(r)) \talque p' \in P\}$
        
        \item Largo $L$ de la cinta. El largo de la cinta dependerá de el conjunto de desplazamientos $V = \{v_1, ..., v_n\}$ y el conjunto de rotaciones $R = \{r_1, ..., r_n\}$, los cuales provienen del conjunto $P$ de polígonos posicionados en la cinta. De este modo, se puede definir $L = L(V, R) = \text{max}( \{p_x \talque (p_x, p_y) \in P_r(r_i) \oplus v_i, \text{ } P_i \in \mathcal{P} \})$.
    \end{itemize}

    \item Restricciones:
    \begin{itemize}
        \item No hay ningún par de polígonos que se superpongan entre ellos:\\ $(P_i(r_i) \oplus v_i) \bigcup (P_j(r_j) \oplus v_j) = \phi; \text{ } 1 \leq i, j \leq n$
        
        \item Todos los polígonos se encuentran completamente dentro de la cinta:\\
        $(P_i(r_i) \oplus v_i) \subseteq \mathcal{C}(W, L); \text{ } 1 \leq i \leq n$
    \end{itemize}
\end{itemize}



\section{Representación}

Representación matemática y estructura de datos que se usa (arreglos, matrices, etc.), por qué se usa (maneja restricciones, fácil de modificar, eficiente, etc), la relación entre la representación matemática y la estructura.


\section{Descripción del algoritmo}

\textcolor{red}{Cómo fue implementando, interesa la implementación más que el algoritmo genérico, es decir, si se tiene que implementar SA, lo que se espera es que se explique en pseudo código la estructura general y en párrafo explicativo cada parte como fue implementada para su caso particular, si se utilizan operadores se debe explicar por que se utilizó ese operador, si fuera el caso de una técnica completa, si se utiliza recursión o no, etc. Use diagramas para mostrar la estructura general del algoritmo, diagramas de flujo de movimientos, esquemas, etc. En este punto no se espera que se incluya código, eso va aparte en la entrega del código fuente.}


\section{Experimentos}

% \textcolor{red}{Se necesita saber cómo se hicieron los experimentos para testear los resultados del algoritmo (metodología usada, entorno de esperimentación, etc.), cuáles son, cómo se definen y cómo se obtienen parámetros del algoritmo, como los fueron modificando, describir las instancias que se usaron (complejidad, estructura, etc), criterio de término (si aplica). Debe comparar su algoritmo con el estado del arte, además de comparar ejecuciones con distintas especificaciones de su mismo algoritmo (Ejm. el valor del parámetro x siendo 0.1 vs 0.5 vs 0.9). Describir cantidad de ejecuciones usando semillas distintas para generar estadísticas.}


El algoritmo fue probado con 12 instancias\footnote{\url{https://github.com/AngheloAlf/2020-2_IA_Milk_collection_problem_with_blending/tree/master/instances}} de distintos tamaños. Además, se le agregó colores al output final del programa para poder saber más fácilmente si el resultado final es factible o no y para saber que calidad de leche produce cada granja asignada a cada ruta. 

Si la solución final es infactible, se muestra en rojo el beneficio total de esta solución y la letra de la ruta infactible. Si la solución no es factible debido a que una ruta no cumple con la cuota mínima de ese tipo de leche, se muestra en rojo el número que indica la cantidad total de leche que se está llevando de este tipo de esa ruta. Además, se le asignó un color a las granjas dependiendo del tipo de leche que producen, si la granja es de tipo 'A' se mostrará en color verde; si es de tipo 'B' se mostrará en color amarillo y así sucesivamente.

Esta metodología permitió detectar fácilmente que se generaban soluciones infactibles, analizar los datos y confeccionar nuevos movimientos y modificar los ya existentes para mejorar la factibilidad de las soluciones.

El entorno de experimentación consiste en un equipo con procesador Intel Core i5-6400 2.70GHz, con 24GB de memoria RAM a 2133MHz y un disco de duro de 2TB y 5400rpm. El equipo cuenta con Pop!\_OS 20.10.

El algoritmo requiere que se le entregue como parámetro los datos de la instancia y el máximo de iteraciones a realizar (llámese $K$).

Para poder ejecutar el programa, se le debe entregar como parámetro la ruta a un archivo que contiene los datos de la instancia y el número entero $K$.

Este archivo indíca la siguiente información:
\begin{itemize}
    \item La cantidad de camiones y la capacidad de cada uno.
    \item La cantidad de tipos de leche, la cuota mínima que se requiere de cada tipo y la ganancia que aporta cada uno.
    \item La cantidad de granjas y la información asociada a cada uno. Se indica un identificador numérico, posiciones x e y, una letra indicando la calidad de esta y la cantidad de leche producida. Cabe destacar que el primero de esta lista no es una granja, si no que se refiere a la planta procesardora.
\end{itemize}

El programa usa los datos de este archivo como parámetro para el algoritmo.

Se utilizaron 12 instancias distintas, las cuales todas tenían 3 camiones y 3 tipos de leche en cada instancia (variando la cuota pedida y la capacidad de los camiones). La diferencia principal de las instancias son en la cantidad de nodos, variando desde 22 hasta 80 nodos, donde las instancias con mayor cantidad de nodos requerían una mayor cantidad de iteraciones para terminar.

El algoritmo tiene 2 criterios de término. El primero es si la cantidad de iteraciones sobrepasa al parámetro $K$, entendiendose una iteración como la realización de alguno de los movimientos propuestos en la sección anterior. El segundo criterio de parada consiste en la detección de un óptimo local, de modo que ninguno de los movimientos propuestos puede mejorar la calidad de la solución.

Todas las instancias fueron probadas repetidas veces con un $K=1000$, pero ninguna ejecución sobrepasó las 500 iteraciones. Normalmente oscilaban entre las 60 (instancias pequeñas) y las 460 iteraciones (instancias más grandes).


\section{Resultados}

Que fue lo que se logró con la experimentación, incluir tablas y parámetros, gráficos (por ejm boxplot), lo más explicativo posible. En los resultados se espera que concluya cuál fue el rendimiento del algoritmo con los experimentos detallados en la sección anterior, y compare las diferencias entre configuraciones distintas de los experimentos. Analizar los resultados obtenidos y concluir acerca de aspectos del algoritmo y/o de la complejidad de las instancias, o acerca de características relacionadas con su implementación.


\section{Conclusiones}

Como cualquier otro problema NP-duro, este problema CSOP ha demostrado ser difícil de modelar de modo que entregue una solución óptima en un tiempo prudente.

Debido a que es un problema bastante joven (aunque VRP data de 1959 \cite{TruckDispatchingProblem}, esta variante especifica se puede encontrar en la literatura desde el año 2016 \cite{MilkWithBlending}) se han desarrollado escasos modelos y métodos de resolución para este problema específico.

De todas formas, las técnicas que se han usado para trabajar \textit{Milk Collection with Blending} suelen atacar al mismo problema base, pero con leves diferencias (como los tamaños de las instancias, o la opción de tener múltiples compartimientos en los camiones). Se han usado tanto técnicas completas, las cuales han demostrado ser viables para instancias pequeñas a medianas, algoritmos basados en técnicas completas y técnicas incompletas, como lo son \textit{hill-climbing} e \textit{iterated local search}. Estás últimas han demostrado mejores resultados que las técnicas completas

Una variante a considerar para investigaciones futuras podría ser la incorporación de múltiples plantas de procesado de leche en lugar de una única planta central.



// De acuerdo a la introducci\'on que se hizo, entregar afirmaciones RELEVANTES basadas en los experimentos y sus resultados. Incluir: Conclusiones sobre el problema, an\'alisis de los resultados, an\'alisis de la t\'ecnica usada, qu\'e fall\'o, qu\'e se podr\'ia mejorar, trabajo futuro que se podr\'ia realizar.



\section{Bibliografía}

\bibliographystyle{plain}
\bibliography{Referencias}

\end{multicols}

\end{document}
