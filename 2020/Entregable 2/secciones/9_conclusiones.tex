\section{Conclusiones}

Como cualquier otro problema NP-duro, este problema CSOP ha demostrado ser difícil de modelar de modo que entregue una solución óptima en un tiempo prudente.

Debido a que es un problema bastante joven (aunque VRP data de 1959 \cite{TruckDispatchingProblem}, esta variante especifica se puede encontrar en la literatura desde el año 2016 \cite{MilkWithBlending}) se han desarrollado escasos modelos y métodos de resolución para este problema específico.

De todas formas, las técnicas que se han usado para trabajar \textit{Milk Collection with Blending} suelen atacar al mismo problema base, pero con leves diferencias (como los tamaños de las instancias, o la opción de tener múltiples compartimientos en los camiones). Se han usado tanto técnicas completas, las cuales han demostrado ser viables para instancias pequeñas a medianas, algoritmos basados en técnicas completas y técnicas incompletas, como lo son \textit{hill-climbing} e \textit{iterated local search}. Estás últimas han demostrado mejores resultados que las técnicas completas

Una variante a considerar para investigaciones futuras podría ser la incorporación de múltiples plantas de procesado de leche en lugar de una única planta central.



// De acuerdo a la introducción que se hizo, entregar afirmaciones RELEVANTES basadas en los experimentos y sus resultados. Incluir: Conclusiones sobre el problema, análisis de los resultados, análisis de la técnica usada, qué falló, qué se podría mejorar, trabajo futuro que se podría realizar.


