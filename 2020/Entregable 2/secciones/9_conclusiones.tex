\section{Conclusiones}

Como cualquier otro problema NP-duro, este problema CSOP ha demostrado ser difícil de modelar de modo que entregue una solución óptima en un tiempo prudente.

Debido a que es un problema bastante joven (aunque VRP data de 1959 \cite{TruckDispatchingProblem}, esta variante especifica se puede encontrar en la literatura desde el año 2016 \cite{MilkWithBlending}) se han desarrollado escasos modelos y métodos de resolución para este problema específico.

De todas formas, las técnicas que se han usado para trabajar \textit{Milk Collection with Blending} suelen atacar al mismo problema base, pero con leves diferencias (como los tamaños de las instancias, o la opción de tener múltiples compartimientos en los camiones). Se han usado tanto técnicas completas, las cuales han demostrado ser viables para instancias pequeñas a medianas, algoritmos basados en técnicas completas y técnicas incompletas, como lo son \textit{hill-climbing} e \textit{iterated local search}. Estás últimas han demostrado mejores resultados que las técnicas completas

Una variante a considerar para investigaciones futuras podría ser la incorporación de múltiples plantas de procesado de leche en lugar de una única planta central.



%\textcolor{red}{De acuerdo a la introducción que se hizo, entregar afirmaciones RELEVANTES basadas en los experimentos y sus resultados. Incluir: Conclusiones sobre el problema, análisis de los resultados, análisis de la técnica usada, qué falló, qué se podría mejorar, trabajo futuro que se podría realizar.}

El algoritmo propuesto entregar soluciones de calidad en un tiempo más que razonable con las instancias utilizadas.

Durante el periodo de pruebas, ninguna de las instancias requirió más de 500 iteraciones para encontrar el óptimo local. Esto lleva a la sospecha de que el algoritmo de hill climbing se está estancando demasiado rápido en óptimos locales. Este estancamiento se puede explicar por la naturaleza inherente a este algoritmo, debido a que este está diseñado de modo que solo explota la solución con la que está trabajando, casi sin explorar otras posibilidades.

