\section{Descripción del algoritmo}

\textcolor{red}{Cómo fue implementando, interesa la implementación más que el algoritmo genérico, es decir, si se tiene que implementar SA, lo que se espera es que se explique en pseudo código la estructura general y en párrafo explicativo cada parte como fue implementada para su caso particular, si se utilizan operadores se debe explicar por que se utilizó ese operador, si fuera el caso de una técnica completa, si se utiliza recursión o no, etc. Use diagramas para mostrar la estructura general del algoritmo, diagramas de flujo de movimientos, esquemas, etc. En este punto no se espera que se incluya código, eso va aparte en la entrega del código fuente.}

Este algoritmo al estar basado en una técnica incompleta como lo es Hill Climbing, está dividido en varios pasos, comenzando con la generación de la solución inicial.

\subsection{Generación de solución inicial}



\subsection{Función de evaluación}



\subsection{Hill climbing}




\subsubsection{Move node between routes}


\subsubsection{2opt intra-route}


\subsubsection{Remove node}


\subsubsection{Interchange node between routes}

