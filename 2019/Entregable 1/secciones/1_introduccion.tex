\section{Introducción}

El presente documento tiene como propósito presentar y definir el problema \textit{Irregular Strip Packing} (el cual consiste en la colocación de figuras irregulares en una cinta de ancho fijo de modo que se minimice el largo de esta cinta), y documentar el actual estado del arte de este problema. 

Se empezará definiendo el problema actual, hablando de forma general de las variables y restricciones típicas de este problema. Luego se expondrá el actual estado del arte del \textit{Irregular Strip Packing Problem} (ISPP), documentando métodos usados para resolverlo, las mejores representaciones y algoritmos hasta la fecha, heurísticas, tendencias y lo más importante que se ha hecho hasta ahora con relación al problema. Se presentará un modelo matemático simple, el cual es capaz de modelar el problema. Se opto por presentar un modelo simple a modo introductorio a los posibles modelos de este problema, con el inconveniente de que este modelo no es eficiente en comparación a otros. Finalmente se expondrán las conclusiones de este trabajo investigativo.

